% Use only LaTeX2e, calling the article.cls class and 12-point type.

\documentclass[12pt]{article}

% Users of the {thebibliography} environment or BibTeX should use the
% scicite.sty package, downloadable from *Science* at
% www.sciencemag.org/about/authors/prep/TeX_help/ .
% This package should properly format in-text
% reference calls and reference-list numbers.

\usepackage{scicite}

% Use times if you have the font installed; otherwise, comment out the
% following line.

\usepackage{times}

% The preamble here sets up a lot of new/revised commands and
% environments.  It's annoying, but please do *not* try to strip these
% out into a separate .sty file (which could lead to the loss of some
% information when we convert the file to other formats).  Instead, keep
% them in the preamble of your main LaTeX source file.


% The following parameters seem to provide a reasonable page setup.

\topmargin 0.0cm
\oddsidemargin 0.2cm
\textwidth 16cm 
\textheight 21cm
\footskip 1.0cm


%The next command sets up an environment for the abstract to your paper.

\newenvironment{sciabstract}{%
\begin{quote} \bf}
{\end{quote}}


% If your reference list includes text notes as well as references,
% include the following line; otherwise, comment it out.

\renewcommand\refname{References and Notes}

% The following lines set up an environment for the last note in the
% reference list, which commonly includes acknowledgments of funding,
% help, etc.  It's intended for users of BibTeX or the {thebibliography}
% environment.  Users who are hand-coding their references at the end
% using a list environment such as {enumerate} can simply add another
% item at the end, and it will be numbered automatically.

\newcounter{lastnote}
\newenvironment{scilastnote}{%
\setcounter{lastnote}{\value{enumiv}}%
\addtocounter{lastnote}{+1}%
\begin{list}%
{\arabic{lastnote}.}
{\setlength{\leftmargin}{.22in}}
{\setlength{\labelsep}{.5em}}}
{\end{list}}


% Include your paper's title here

\title{Copy Move Detection} 


% Place the author information here.  Please hand-code the contact
% information and notecalls; do *not* use \footnote commands.  Let the
% author contact information appear immediately below the author names
% as shown.  We would also prefer that you don't change the type-size
% settings shown here.

\author
{Anthony Sutardja, Kevin Tee\\
\\
\normalsize{Department of Computer Science}\\
\normalsize{University of California, Berkeley}\\
\normalsize{CS294-26 Final Project}\\
\\
\normalsize\texttt{\{anthonysutardja,kevintee\}@berkeley.edu} 
}

% Include the date command, but leave its argument blank.

\date{}



%%%%%%%%%%%%%%%%% END OF PREAMBLE %%%%%%%%%%%%%%%%



\begin{document} 

% Double-space the manuscript.

\baselineskip24pt

% Make the title.

\maketitle 



% Place your abstract within the special {sciabstract} environment.

\begin{sciabstract}
  Abstract: Copy Move Detection...
\end{sciabstract}



% In setting up this template for *Science* papers, we've used both
% the \section* command and the \paragraph* command for topical
% divisions.  Which you use will of course depend on the type of paper
% you're writing.  Review Articles tend to have displayed headings, for
% which \section* is more appropriate; Research Articles, when they have
% formal topical divisions at all, tend to signal them with bold text
% that runs into the paragraph, for which \paragraph* is the right
% choice.  Either way, use the asterisk (*) modifier, as shown, to
% suppress numbering.

\section*{Introduction}

In this paper, ...

\section*{Methodology}

Our methods...

\section*{Results}

Our results...

\begin{table}[t]
\caption{SAMPLE: Topics Derived From Sparse SVD, n = 7}
\label{sparsesvd}
\begin{center}
\begin{tabular}{lllllll}
\multicolumn{1}{l}{\bf Topic 1} & \multicolumn{1}{l}{\bf Topic 2} & \multicolumn{1}{l}{\bf Topic 3} & \multicolumn{1}{l}{\bf Topic 4} & \multicolumn{1}{l}{\bf Topic 5} & \multicolumn{1}{l}{\bf Topic 6} & \multicolumn{1}{l}{\bf Topic 7}
\\ \hline \\
Silicon & Input & Polypeptides & Spindle & Ester & Freely & Data \\
Metal & Frequency & Tissues & Lens & Fibers & Branching & System \\
Formed & Valve & Fibers & Motor & Keto & Foliage & Fiber \\
Electrode & Amplifier & Methods & Gas & Acid & Flowers & Information \\
Surface & Power & Compounds & Optical & Valve & Yellow & Network \\
Semiconductor & Current & Products & Drive & Tissue & Plant & Signal \\
Substrate & Circuit & Ink & Magnetic & Crosslinked & Habit & Optical \\
\end{tabular}
\end{center}
\end{table}

%*********************SAMPLE FIGURE*************************
%\begin{figure}[h]
%\begin{center}
%\includegraphics[width=4in]{figure_1.png}
%\end{center}
%\caption{Accuracy of Each Algorithms}
%\end{figure}

\section*{Discussion and Future Work}

Discuss here...

\section*{Acknowledgements}

Acknowledge here...

\section*{References}

SAMPLE

\begin{enumerate}
\item G. Gamow, {\it The Constitution of Atomic Nuclei and
Radioactivity\/} (Oxford Univ. Press, New York, 1931).
\item W. Heisenberg and W. Pauli, {\it Zeitschr.\ f.\ Physik} {\bf 56},
1 (1929).
\end{enumerate}

\end{document}




















